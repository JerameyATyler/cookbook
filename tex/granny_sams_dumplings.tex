
\begin{recipe}
[% 
    preparationtime = {\unit[1]{h} - Half the day (if you do it with love)},
    source = {Granny Sams' Cookbook}
]
{Granny Sams' Dumplins}
    
    \graph
    {% pictures
        small=pic/glass,     % small picture
        big=pic/ingredients  % big picture
    }
    
    \introduction{%
        Greeting Taste Travellers! You're in for a special treat today. My great (great?) grandmother had this dumpling recipe in an old cookbook from her church. This is how I recieved it. I'll add proper "stock" recipe later, likely as a whole chicken'n Dumplins recipe. 
    }
    
    \ingredients)\\
        1/2 tsp & baking powder (must be fresh and not expired)\\
        1/2 tsp & Salt\\
        1 stick & Cold Butter (cut into small pieces)\\
        1 & egg (lightly beaten in small coffee cup)\\
        50 g & water added to beaten egg and mix well
    }
    
    \preparation{%
        \step Mix dry ingredients well. Add butter to dry ingredients and cut with pastry 					  cutter till small crumbs form (several minutes). Add egg/water mixture and mix 		      well. Dough will be very sticky.

        \step On floured surface, roll out dough (add flour as needed) turning over 						  occasionally while rolling dough to about 1/4 to 1/8 inch thick (dough will no 			  longer be sticky and looks dry on the surface). Cut into squares about 2"x2". 				  Should make about 100 dumpling squares. 

        \step Drop into boiling stock pot, pushing to the bottom. Use all the dough 						  including odd or thin pieces (those will disappear and thicken the stock). 
        
        \step Once all pieces are added, reduce heat to simmer, cover and cook about 45 			          minutes (stirring occasionally) or till done inside (taste testing is best way 			  to tell). Most times they will all sink to the bottom but be sure to taste 					  first.  
    }    
\end{recipe}